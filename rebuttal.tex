%\documentclass{acmart}
%\documentclass{amsart}
\documentclass{article}
%\documentclass{acmart}
\usepackage[utf8]{inputenc}


% packages 
\usepackage[ruled,vlined]{algorithm2e}
\usepackage[titles]{tocloft}%----------------------------------------
\usepackage{mathrsfs}
\usepackage{bm}
\usepackage{enumitem}
%% \usepackage{enumerate}
\usepackage{chngcntr}
\usepackage{titlesec}
\usepackage{mathrsfs}
\usepackage{bm}
\usepackage{enumitem}
\usepackage{amsmath}
\usepackage{amssymb}
\usepackage{amsthm}
\usepackage{geometry}
 \geometry{
 a4paper,
 total={170mm,257mm},
 left=20mm,
 top=20mm,
 }

% new commands and shortcuts


\def\sO{\mathscr{O}}
\def\gip{\Gamma_i^{'}}
\def\gi{\Gamma_i}
\def\td{{\bf todo}}
\def\bs{\textit{\textbf{s}}}
\def\ub{\textit{\textbf{u}}}
\def\bz{\textit{\textbf{0}}}
\def\Lb{\textit{\textbf{L}}}
\def\Ls{\mathscr{L}}
\def\Xb{\textit{\textbf{X}}}
\def\lb{\textit{\textbf{l}}}
\def\Lambdab{\bm{\Lambda}}
\def\Thetab{\bm{\Theta}}
\def\thetab{\bm{\vartheta}}
\def\mA{{\bm A}}
\def\fA{{\frak A}}
\def\xb{\textit{\textbf{x}}}
\def\fb{\textit{\textbf{f}}}
\def\ab{\textit{\textbf{a}}}
\def\pb{\textit{\textbf{p}}}
\def\ajb{\overline{a_j}}
\def\bjb{\overline{b_{j,l}}}
\def\pjb{\overline{P_{j}}}
\DeclareMathOperator{\GL}{GL}
\DeclareMathOperator{\crit}{crit}
\DeclareMathOperator{\htt}{ht}
\def\C{\mathbb{C}}
\def\pr{\mathbb{P}}
\def\vt{\vartheta}
\def\Q{\mathbb{Q}}
\def\N{\mathbb{N}}
\def\R{\mathbb{R}}
\def\K{\mathbb{K}}
\def\rs{\mathscr{R}}
\def\P{\mathscr{P}}
\def\d{\delta}
\def\sing{ \textrm{sing}}
\def\codim{ \textrm{codim}}
\DeclareMathOperator{\jac}{jac}
\def\exp{\textrm{exp}}
\def\grad{\textbf{grad}}
\def\rank{\textrm{rank}}
\def\reg{\textrm{reg}}
\def\jt{\widetilde{J}}
\def\At{\widetilde{A}}
\def\Yt{\widetilde{Y}}
\def\dtt{\widetilde{d}}
\def\kt{\widetilde{k}}
\def\dt{s}
\def\bI{\textbf{I}}
%% \def\dt{\widetilde{d}}
\def\Dt{\widetilde{D}}
\def\rk{\textrm{rank }}
\def\pa{\partial}
\def\D{\Delta}
\def\Z{\frak{Z}}
\newcommand{\ZZ}{{\mathbb{Z}}}
\newcommand{\softO}{{O^{\sim}}}
\def\I{\frak{I}}
\def\Is{\frak{I}^{\star}}
\def\A{\frak{A}}
\def\fp{\frak{P}}
\def\sp{\mathscr{P}}
\def\la{\langle}
\def\ra{\rangle}
% matrices 
\def\scrQ{\ensuremath{\mathscr{Q}}}
\def\bbm{\begin{bmatrix}}
\def\ebm{\end{bmatrix}}
%%
% Theorems 
\newtheorem{theorem}{Theorem}
\newtheorem{corollary}[theorem]{Corollary}
\newtheorem{lemma}[theorem]{Lemma}
\newtheorem{observation}[theorem]{Observation}
\newtheorem{prop}[theorem]{Proposition}
\newtheorem{definition}[theorem]{Definition}
\newtheorem{claim}[theorem]{Claim}
\newtheorem{fact}[theorem]{Fact}
\newtheorem{assumption}[theorem]{Assumption}
\newtheorem{remark}[theorem]{Remark}
\newtheorem{question}{Question}
%\newtheorem*{ex}{Example}
\newtheorem{cor}[theorem]{Corollary}







\begin{document}

\title{Answer to the reviews for Submission 98}

\author{}
\date{}


\maketitle

\noindent \textbf{Responding to questions.} We take
$f\in\ZZ[X_1\hdots,X_n]$ squarefree, satisfying $\deg(f) \leq d$ and
$V=V(f) \subset \C^n$ smooth. For $i \in \{1,\hdots,n-1\},$ denote by
$\pi_i:\C^n \rightarrow \C^i$ the projection $(x_1,\hdots,x_n) \mapsto
(x_1,\hdots,x_i)$.  The $i$-th \textit{polar variety} \[W(\pi_i,V) :=
\{\xb \in V~|~\dim \pi_i(T_\xb V) < i\}\] is the set of critical
points of $\pi_i$ on $V$.
For a matrix $\mA \in \C^{n \times n}$ we define $f^\mA := f(\mA\xb),$
and for a variety $Y \subset \C^n,$ $Y^{\mA}$ is the image of $Y$ by
the map $\phi_{\mA} : \xb \mapsto \mA^{-1}\xb.$

There is some overlap between the questions in the reviews. To
avoid repetition, we do not respond to each review individually.

\begin{enumerate}
    \item \textit{Can we replace each upper degree bound $d^n$ by
      $\deg W(\pi_i,V)$? }
    
    This is not straightforward to replace $d^n$; we are not sure how
    to do it. In any case, we do not expect that it would improve the
    worst-case bounds.
    
    \item \textit{Can you explain the FAIL output in the main algorithm?}
    
    The main algorithm is guaranteed to succeed as long as our call
    to Algorithm 2 in [30] does. The latter is randomized; when it
    does not succeed, it either returns a proper subset of the
    solutions, or FAIL.
        
    \item\textit{Is the quantitative Noether position statement a new result? }
    
    In Theorem 2.1, we show that, for $i=1,\dots,n$, there exists a
    non-zero polynomial $\D_i$ of degree at most
    $6n^2(2d)^{2n}$ such that if $\mA \in \C^{n\times n}$ does not
    cancel $\D_i$, then $\mA$ is invertible and $W(\pi_i,V(f^{\mA}))$
    is either empty or in Noether position for $\pi_{i-1}$.

    In the second review, it was pointed out that Lemma 5 in G. Jeronimo
    and J. Sabia, {\em Effective equidimensional decomposition of affine
      varieties} or Proposition 4.5 in T. Krick, L.-M. Pardo and
    M. Sombra, {\em Sharp estimates for the arithmetic Nullstellensatz},
    are quantitative Noether position statements. The question is  
    whether our Theorem 2.1 is new. 
    
    Our theorem does not follow from these previous results. Indeed,
    results in the references above would allow us to quantify when
    $W(\pi_i,V(f))^\mA$ is in Noether position, whereas we need to
    understand when $W(\pi_i,V(f^\mA))$ is. These two sets are in general
    different: the operations of computing the critical points of a
    projection and changing variables do not commute (this can be seen
    already for polynomials such as $f = X_1^2 + 2X_2^2-1$).
    We should have made it clearer.
\end{enumerate}

\vspace{2mm}
\par 
\noindent \textbf{On the student author's contribution.}  Jesse
Elliott is a Master's student, under the supervision of Mark
Giesbrecht and \'Eric Schost; this paper will form the basis of his
Master's thesis. Jesse was the main contributor to this submission. He
came up with the idea of analyzing the bit-complexity of a former
algorithm by Safey El Din and Schost, worked out most estimates, and
wrote the bulk of the paper. He also wrote this rebuttal (except of
course for this paragraph). We (MG and \'ES) would like to nominate
him for a distinguished student paper award.








\bibliographystyle{plain}
\bibliography{sample-base}
\end{document}
